\documentclass[11pt,a4paper]{article}

% Packages
\usepackage{amsmath,amssymb,amsthm,mathtools}
\usepackage[utf8]{inputenc}
\usepackage[margin=0.75in]{geometry}
\usepackage{enumerate}
\usepackage{hyperref}

% Theorem environments
\newtheorem{theorem}{Theorem}
\newtheorem{lemma}[theorem]{Lemma}
\newtheorem{proposition}[theorem]{Proposition}
\newtheorem{corollary}[theorem]{Corollary}
\theoremstyle{definition}
\newtheorem{definition}[theorem]{Definition}
\newtheorem{example}[theorem]{Example}
\theoremstyle{remark}
\newtheorem{remark}[theorem]{Remark}

% Custom commands
\newcommand{\R}{\mathbb{R}}
\newcommand{\N}{\mathbb{N}}
\newcommand{\Z}{\mathbb{Z}}
\newcommand{\Q}{\mathbb{Q}}
\newcommand{\C}{\mathbb{C}}

% Document info
\title{Assignment Title}
\author{Firstname Lastname \\ Course Name}
\date{\today}

\begin{document}

\maketitle

\section{Problem 1}

State the problem here.

\textbf{Solution.} Provide your solution with inline math $x^2 + y^2 = r^2$ or display math:
\[
\int_{-\infty}^{\infty} e^{-x^2} \, dx = \sqrt{\pi}
\]

\begin{theorem}
Every positive integer greater than 1 can be written uniquely as a product of prime numbers.
\end{theorem}

\begin{proof}
The proof goes here.
\end{proof}

Aligned equations:
\begin{align}
f(x) &= x^2 + 2x + 1 \\
     &= (x + 1)^2 \label{eq:example}
\end{align}

Reference equation \eqref{eq:example} using labels.

\section{Problem 2}

\begin{definition}
A \textbf{group} is a set $G$ with operation $\cdot$ satisfying associativity, identity, and inverses.
\end{definition}

\begin{example}
The integers $\Z$ with addition form a group.
\end{example}

\end{document}
