\documentclass{article}

% Packages
\usepackage[a4paper]{geometry}
\usepackage[german]{babel}
\usepackage[T1]{fontenc}
\usepackage{amsmath,amssymb,amsthm,mathtools}
\usepackage[utf8]{inputenc}
\usepackage{fancyhdr}
\usepackage{titlesec}
\usepackage{enumerate}

% Theorem environments
\newtheorem{theorem}{Theorem}
\newtheorem{lemma}[theorem]{Lemma}
\newtheorem{proposition}[theorem]{Proposition}
\newtheorem{corollary}[theorem]{Corollary}
\newtheorem{definition}[theorem]{Definition}
\newtheorem{example}[theorem]{Example}
\newtheorem{remark}[theorem]{Remark}

\theoremstyle{definition}
\theoremstyle{remark}

% Fancy header
\pagestyle{fancyplain}
\fancyhf{}
\lhead{\fancyplain{}{Firstname Lastname} }
\rhead{\fancyplain{}{\today}}
\cfoot{\fancyplain{}{\thepage}}

% Title format
\titleformat{\section}{\normalfont\large\bfseries}{\thesection}{1em}{}
\titleformat{\subsection}{\normalfont\normalsize\bfseries}{\thesubsection}{1em}{}
\titleformat{\subsubsection}{\normalfont\small\bfseries}{\thesubsubsection}{1em}{}
\titleformat{\paragraph}{\normalfont\small\bfseries}{\theparagraph}{1em}{}
\titlespacing*{\paragraph}{0pt}{3.25ex plus 1ex minus .2ex}{1.5ex plus .2ex}

% Set counter
\setcounter{section}{1}

% Set date
\date{\today}

% Document info
\title{Title}
\author{Firstname Lastname \\ Subject}

\begin{document}

\maketitle

\section{Problem 1}

State the problem here.

\textbf{Solution.} Provide your solution with inline math $x^2 + y^2 = r^2$ or display math:
\[
\int_{-\infty}^{\infty} e^{-x^2} \, dx = \sqrt{\pi}
\]

\begin{theorem}
Every positive integer greater than 1 can be written uniquely as a product of prime numbers.
\end{theorem}

\begin{proof}
The proof goes here.
\end{proof}

Aligned equations:
\begin{align}
f(x) &= x^2 + 2x + 1 \\
     &= (x + 1)^2 \label{eq:example}
\end{align}

Reference equation \eqref{eq:example} using labels.

\section{Problem 2}

\begin{definition}
A \textbf{group} is a set $G$ with operation $\cdot$ satisfying associativity, identity, and inverses.
\end{definition}

\end{document}
